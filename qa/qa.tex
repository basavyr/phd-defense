\documentclass[a4paper,12pt]{article}

\usepackage{amsmath}
\usepackage{xcolor}
\usepackage{physics}
\usepackage{graphicx}


\author{Robert Poenaru}
\date{\today}
\title{Phd Deference Preparation\\Questions}
\begin{document}

\maketitle

\tableofcontents

\section{Set 1}

\begin{center}
    \textit{As of 01.05.2023}
\end{center}

\begin{center}
\textbf{Pe baza articolului din 2020 cu descrierea bozonica}
\end{center}

\begin{enumerate}
    \item In expresia initiala a Hamiltonianului de rotator:
    \begin{align}
        \hat{H}_\text{rot}=A\hat{H}'+(A_1I^2-A_2j_2I)+\sum_{k=1,2}A_k\hat{j}_k^2\ ,
    \end{align}
    , care termen \emph{inglobeaza} efectul fortei Coriolis asupra sistemului de rotor + particula?
    \item Este intuitiv faptul ca apare o constrangere asupra particulei imparte sa se miste liber doar in jurul axelor 1-2 din aceasta expresie?
    \item Coordonata $q$ cu care este construit matematic si apoi reprezentat potentialul "eliptic" $V(q)$ poate fi, in cele din urma, considerata ca fiind legata de deformarea triaxiala propriu-zisa a nucleului? (i.e., sa fie cumva legata de coordonatele $\beta$ si $\gamma$, avand in vedere ca $q$ se construieste din $k$ si, la randul lui din $u$)
    \item Item daca nu, care ar putea fi o interpretare fizica strict al triaxialitatii pentru $q$ ?
    \item Pentru a obtine formula frecventei de wobbling $\omega$ din Ecuatia (4.4) din articol, s-a rezolvat ecuatia lui Schrodiner folosind potentialul $V(q)$ expandat in serie de puteri pana la $q^2$. Ar putea sa ma intrebe comisia daca termeni in $q^3$ nu ar putea fi relevanti?
\end{enumerate}


\section{Set 2}
\textit{...}

\end{document}